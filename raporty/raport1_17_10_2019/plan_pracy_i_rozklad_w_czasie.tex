\section{Plan pracy i rozkład w czasie}
\label{sec:plac_pracy_kamienie_milowe_i_rozklad_w_czasie} %definicja etykiety rozdzialu
\subsection{Plan pracy}
\begin{table}[h!]
\begin{center}
\begin{tabular}{|p{6cm}|p{2.2cm}|p{2.2cm}|p{3.2cm}|} % deklaracja tabeli złożonej z 4 kolumn
\hline
Nazwa zadania & Data rozpoczęcia & Data zakończenia & Wykowawca \\ \hline
Rozpoznanie projektowe & 2019-10-03 & 2019-10-10 & Cała grupa\\ \hline
Opracowanie założeń & 2019-10-10 & 2019-10-17 & Krzysztof Danielak, Jakub Taczała, Wojciech Kosicki\\ \hline
Analiza dostępnych środowisk & 2019-10-10 & 2019-10-17 & Piotrek Gogola \\ \hline
Opracowanie algorytmu ogólnego & 2019-10-17 & 2019-10-24 & Jakub Taczała\\ \hline
Opracowanie systemu generowania mapy & 2019-10-17 & 2019-10-24 & Piotrek Gogola, Wojciech Kosicki\\ \hline
Opracowanie algorytmu szczegółowego centrali  & 2019-10-24 & 2019-11-14 & Wojciech Kosicki\\ \hline
Opracowanie algorytmu szczegółowego robota roju  & 2019-10-24 & 2019-11-14 & Krzysztof Danielak\\ \hline
Implementacja jednostki centralnej & 2019-11-14 & 2019-11-28 & Piotrek Gogola,  Wojciech Kosicki\\ \hline
Implementacja osobnika roju & 2019-11-14 & 2019-11-28 & Krzysztof Danielak, Jakub Taczała\\ \hline
Scalenie roju z jednostką centralną & 2019-11-28 & 2019-12-05 &  Cała grupa\\ \hline
Testy na pojedynczym osobniku & 2019-12-05 & 2019-12-12 &  Cała grupa\\ \hline
Testy na wielu osobnikach & 2019-12-12 & 2019-12-20 &  Cała grupa\\ \hline
Wprowadzenie poprawek & 2019-12-20 & 2020-01-09 &  Cała grupa\\ \hline
Stworzenie ostatecznej dokumentacji projektowej & 2020-01-09 & 2020-01-16 &  Cała grupa\\ \hline
\end{tabular}.
\end{center}
\caption{\label{rj}Zadania} % Nadanie etykiety
% tablicy (\label{rj}) i jej opis.
\end{table}
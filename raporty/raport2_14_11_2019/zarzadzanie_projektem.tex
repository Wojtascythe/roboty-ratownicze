\section{Zarządzanie projektem}
\label{sec:zarzadzanie_projektem} %definicja etykiety rozdziału
Wszelkie konflikty zaistniałe podczas realizacji projektu będą rozwiązywane poprzez ostateczną decyzję koordynatora projektu po wcześniejszej rozmowie z członkami zespołu. Projekt będzie powstawał pod ciągłą kontrolą postępów pracy, co pozwoli na równoległe rozwiązywanie zaistniałych problemów.

Koordynator zespołu ma możliwość ocenienia pracy każdego z członków poprzez uwzględnienie jego zaangażowania. Zadaniem koordynatora jest przydzielenie poszczególnych zadań, weryfikacja jakości, postępu oraz możliwość modyfikacji w razie takiej konieczności. Rolą koordynatora jest także motywacja, wsparcie mentalne i merytoryczne oraz rozwijanie umiejętności miękkich wśród reszty grupy projektowej.

Postępy prac będą monitorowane poprzez odgórny nadzór koordynatora poprzez raporty ustne, oraz wgląd w postęp prac. Z zebranych informacji koordynator będzie tworzył raporty postępu prac, które będą publikowane za pośrednictwem programu E-portal.pl. Raporty składowane będą na platformie overleaf.com, która pozwala na edycję dokumentów pisanych w składni {\LaTeX} przez wielu użytkowników jednocześnie. Kod źródłowy oprogramowania będzie składowany przez serwis github.com.

Każde zadanie przed realizacja zostanie omówione, oraz zostaną określone cele jego realizacji. Następnie po stworzeniu danego modułu i poddaniu go testom zostanie omówiony wynik, na podstawie którego zostanie stworzony raport. 